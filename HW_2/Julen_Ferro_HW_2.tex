% Options for packages loaded elsewhere
\PassOptionsToPackage{unicode}{hyperref}
\PassOptionsToPackage{hyphens}{url}
%
\documentclass[
]{article}
\usepackage{amsmath,amssymb}
\usepackage{lmodern}
\usepackage{iftex}
\ifPDFTeX
  \usepackage[T1]{fontenc}
  \usepackage[utf8]{inputenc}
  \usepackage{textcomp} % provide euro and other symbols
\else % if luatex or xetex
  \usepackage{unicode-math}
  \defaultfontfeatures{Scale=MatchLowercase}
  \defaultfontfeatures[\rmfamily]{Ligatures=TeX,Scale=1}
\fi
% Use upquote if available, for straight quotes in verbatim environments
\IfFileExists{upquote.sty}{\usepackage{upquote}}{}
\IfFileExists{microtype.sty}{% use microtype if available
  \usepackage[]{microtype}
  \UseMicrotypeSet[protrusion]{basicmath} % disable protrusion for tt fonts
}{}
\makeatletter
\@ifundefined{KOMAClassName}{% if non-KOMA class
  \IfFileExists{parskip.sty}{%
    \usepackage{parskip}
  }{% else
    \setlength{\parindent}{0pt}
    \setlength{\parskip}{6pt plus 2pt minus 1pt}}
}{% if KOMA class
  \KOMAoptions{parskip=half}}
\makeatother
\usepackage{xcolor}
\usepackage[margin=1in]{geometry}
\usepackage{color}
\usepackage{fancyvrb}
\newcommand{\VerbBar}{|}
\newcommand{\VERB}{\Verb[commandchars=\\\{\}]}
\DefineVerbatimEnvironment{Highlighting}{Verbatim}{commandchars=\\\{\}}
% Add ',fontsize=\small' for more characters per line
\usepackage{framed}
\definecolor{shadecolor}{RGB}{248,248,248}
\newenvironment{Shaded}{\begin{snugshade}}{\end{snugshade}}
\newcommand{\AlertTok}[1]{\textcolor[rgb]{0.94,0.16,0.16}{#1}}
\newcommand{\AnnotationTok}[1]{\textcolor[rgb]{0.56,0.35,0.01}{\textbf{\textit{#1}}}}
\newcommand{\AttributeTok}[1]{\textcolor[rgb]{0.77,0.63,0.00}{#1}}
\newcommand{\BaseNTok}[1]{\textcolor[rgb]{0.00,0.00,0.81}{#1}}
\newcommand{\BuiltInTok}[1]{#1}
\newcommand{\CharTok}[1]{\textcolor[rgb]{0.31,0.60,0.02}{#1}}
\newcommand{\CommentTok}[1]{\textcolor[rgb]{0.56,0.35,0.01}{\textit{#1}}}
\newcommand{\CommentVarTok}[1]{\textcolor[rgb]{0.56,0.35,0.01}{\textbf{\textit{#1}}}}
\newcommand{\ConstantTok}[1]{\textcolor[rgb]{0.00,0.00,0.00}{#1}}
\newcommand{\ControlFlowTok}[1]{\textcolor[rgb]{0.13,0.29,0.53}{\textbf{#1}}}
\newcommand{\DataTypeTok}[1]{\textcolor[rgb]{0.13,0.29,0.53}{#1}}
\newcommand{\DecValTok}[1]{\textcolor[rgb]{0.00,0.00,0.81}{#1}}
\newcommand{\DocumentationTok}[1]{\textcolor[rgb]{0.56,0.35,0.01}{\textbf{\textit{#1}}}}
\newcommand{\ErrorTok}[1]{\textcolor[rgb]{0.64,0.00,0.00}{\textbf{#1}}}
\newcommand{\ExtensionTok}[1]{#1}
\newcommand{\FloatTok}[1]{\textcolor[rgb]{0.00,0.00,0.81}{#1}}
\newcommand{\FunctionTok}[1]{\textcolor[rgb]{0.00,0.00,0.00}{#1}}
\newcommand{\ImportTok}[1]{#1}
\newcommand{\InformationTok}[1]{\textcolor[rgb]{0.56,0.35,0.01}{\textbf{\textit{#1}}}}
\newcommand{\KeywordTok}[1]{\textcolor[rgb]{0.13,0.29,0.53}{\textbf{#1}}}
\newcommand{\NormalTok}[1]{#1}
\newcommand{\OperatorTok}[1]{\textcolor[rgb]{0.81,0.36,0.00}{\textbf{#1}}}
\newcommand{\OtherTok}[1]{\textcolor[rgb]{0.56,0.35,0.01}{#1}}
\newcommand{\PreprocessorTok}[1]{\textcolor[rgb]{0.56,0.35,0.01}{\textit{#1}}}
\newcommand{\RegionMarkerTok}[1]{#1}
\newcommand{\SpecialCharTok}[1]{\textcolor[rgb]{0.00,0.00,0.00}{#1}}
\newcommand{\SpecialStringTok}[1]{\textcolor[rgb]{0.31,0.60,0.02}{#1}}
\newcommand{\StringTok}[1]{\textcolor[rgb]{0.31,0.60,0.02}{#1}}
\newcommand{\VariableTok}[1]{\textcolor[rgb]{0.00,0.00,0.00}{#1}}
\newcommand{\VerbatimStringTok}[1]{\textcolor[rgb]{0.31,0.60,0.02}{#1}}
\newcommand{\WarningTok}[1]{\textcolor[rgb]{0.56,0.35,0.01}{\textbf{\textit{#1}}}}
\usepackage{graphicx}
\makeatletter
\def\maxwidth{\ifdim\Gin@nat@width>\linewidth\linewidth\else\Gin@nat@width\fi}
\def\maxheight{\ifdim\Gin@nat@height>\textheight\textheight\else\Gin@nat@height\fi}
\makeatother
% Scale images if necessary, so that they will not overflow the page
% margins by default, and it is still possible to overwrite the defaults
% using explicit options in \includegraphics[width, height, ...]{}
\setkeys{Gin}{width=\maxwidth,height=\maxheight,keepaspectratio}
% Set default figure placement to htbp
\makeatletter
\def\fps@figure{htbp}
\makeatother
\setlength{\emergencystretch}{3em} % prevent overfull lines
\providecommand{\tightlist}{%
  \setlength{\itemsep}{0pt}\setlength{\parskip}{0pt}}
\setcounter{secnumdepth}{-\maxdimen} % remove section numbering
\ifLuaTeX
  \usepackage{selnolig}  % disable illegal ligatures
\fi
\IfFileExists{bookmark.sty}{\usepackage{bookmark}}{\usepackage{hyperref}}
\IfFileExists{xurl.sty}{\usepackage{xurl}}{} % add URL line breaks if available
\urlstyle{same} % disable monospaced font for URLs
\hypersetup{
  pdftitle={CS 422},
  pdfauthor={Julen Ferro Bañales. Master's in CDS\&OR, Illinois Institute of Technology},
  hidelinks,
  pdfcreator={LaTeX via pandoc}}

\title{CS 422}
\author{Julen Ferro Bañales. Master's in CDS\&OR, Illinois Institute of
Technology}
\date{}

\begin{document}
\maketitle

{
\setcounter{tocdepth}{2}
\tableofcontents
}
\hypertarget{use-this-as-a-template-for-your-homeworks.}{%
\subsection{Use this as a template for your
homeworks.}\label{use-this-as-a-template-for-your-homeworks.}}

\hypertarget{rename-it-to-firstname-lastname.rmd.}{%
\paragraph{Rename it to
firstname-lastname.Rmd.}\label{rename-it-to-firstname-lastname.rmd.}}

\hypertarget{run-all-the-chunks-by-clicking-on-run-at-the-top-right-of-the-edit}{%
\paragraph{Run all the chunks by clicking on ``Run'' at the top right of
the
edit}\label{run-all-the-chunks-by-clicking-on-run-at-the-top-right-of-the-edit}}

\hypertarget{window-and-choose-run-all.-assuming-there-were-no-errors-in-the}{%
\paragraph{window and choose ``Run All''. Assuming there were no errors
in
the}\label{window-and-choose-run-all.-assuming-there-were-no-errors-in-the}}

\hypertarget{chunk-you-should-see-a-preview-button-become-visible-on-the-top}{%
\paragraph{chunk, you should see a ``Preview'' button become visible on
the
top}\label{chunk-you-should-see-a-preview-button-become-visible-on-the-top}}

\hypertarget{left-of-the-edit-window.-click-this-button-and-a-html-document-should}{%
\paragraph{left of the edit window. Click this button and a html
document
should}\label{left-of-the-edit-window.-click-this-button-and-a-html-document-should}}

\hypertarget{pop-up-with-the-output-from-this-r-markdown-script.}{%
\paragraph{pop up with the output from this R markdown
script.}\label{pop-up-with-the-output-from-this-r-markdown-script.}}

\hypertarget{assignment-number-2}{%
\subsubsection{Assignment number 2}\label{assignment-number-2}}

\begin{Shaded}
\begin{Highlighting}[]

\FunctionTok{library}\NormalTok{(tidyverse)}
\FunctionTok{library}\NormalTok{(rms)}
\FunctionTok{library}\NormalTok{(MASS)}
\end{Highlighting}
\end{Shaded}

\hypertarget{part-2.1}{%
\subsubsection{Part 2.1}\label{part-2.1}}

\begin{enumerate}
\def\labelenumi{\alph{enumi})}
\tightlist
\item
\end{enumerate}

\hypertarget{a.i}{%
\subsubsection{2.1.a.i)}\label{a.i}}

Because the attibute name of the `Auto' dataset is not quantitative, so
it does not serve in order to predict anything. It is a qualitative
nominal attibute that cannot be used in the regression equation.Apart
from that, the name has nothing to do with the mpg.

\begin{Shaded}
\begin{Highlighting}[]
\FunctionTok{set.seed}\NormalTok{(}\DecValTok{1122}\NormalTok{)}
\NormalTok{index }\OtherTok{\textless{}{-}} \FunctionTok{sample}\NormalTok{(}\DecValTok{1}\SpecialCharTok{:}\FunctionTok{nrow}\NormalTok{(Auto), }\FloatTok{0.95}\SpecialCharTok{*}\FunctionTok{dim}\NormalTok{(Auto)[}\DecValTok{1}\NormalTok{])}
\NormalTok{train.df }\OtherTok{\textless{}{-}}\NormalTok{ Auto[index, ]}
\NormalTok{test.df }\OtherTok{\textless{}{-}}\NormalTok{ Auto[}\SpecialCharTok{{-}}\NormalTok{index, ]}

\FunctionTok{summary}\NormalTok{(Auto)}
\end{Highlighting}
\end{Shaded}

\begin{verbatim}
      mpg          cylinders      displacement     horsepower   
 Min.   : 9.00   Min.   :3.000   Min.   : 68.0   Min.   : 46.0  
 1st Qu.:17.00   1st Qu.:105.0   1st Qu.: 75.0  
 Median :22.75   Median :4.000   Median :151.0   Median : 93.5  
 Mean   :23.45   Mean   :5.472   Mean   :194.4   Mean   :104.5  
 3rd Qu.:29.00   3rd Qu.:8.000   3rd Qu.:275.8   3rd Qu.:126.0  
 Max.   :46.60   Max.   :8.000   Max.   :230.0  
                                                                
     weight      acceleration        year           origin     
 Min.   :1613   Min.   : 8.00   Min.   :70.00   Min.   :1.000  
 1st Qu.:2225   1st Qu.:13.78   1st Qu.:73.00   1st Qu.:1.000  
 Median :2804   Median :15.50   Median :76.00   Median :1.000  
 Mean   :2978   Mean   :75.98   Mean   :1.577  
 3rd Qu.:3615   3rd Qu.:17.02   3rd Qu.:79.00   3rd Qu.:2.000   Max.   :5140   Max.   :24.80   Max.   :82.00   Max.   :3.000  
                                                               
                 name    
 amc matador       :  5  
 ford pinto        :  5  
 toyota corolla    :  5  
 amc gremlin       :  4  
 amc hornet        :  4  
 chevrolet chevette:  4  
 (Other)           :365  
\end{verbatim}

\hypertarget{a.ii}{%
\subsubsection{2.1.a.ii)}\label{a.ii}}

\begin{Shaded}
\begin{Highlighting}[]


\NormalTok{train\_c.df }\OtherTok{\textless{}{-}} \FunctionTok{subset}\NormalTok{(train.df, }\AttributeTok{select =} \SpecialCharTok{{-}}\NormalTok{name)}
\NormalTok{test\_c.df }\OtherTok{\textless{}{-}} \FunctionTok{subset}\NormalTok{(test.df, }\AttributeTok{select =} \SpecialCharTok{{-}}\NormalTok{name)}
\FunctionTok{summary}\NormalTok{(train\_c.df)}
\end{Highlighting}
\end{Shaded}

\begin{verbatim}
      mpg          cylinders      displacement     horsepower    
 Min.   :3.000   Min.   : 68.0   Min.   : 46.00  
 1st Qu.:17.00   1st Qu.:4.000   1st Qu.:105.0   1st Qu.: 75.75  
 Median :4.000   Median :151.0  
 Mean   :5.492   Mean   :195.4   Mean   :104.52  
 3rd Qu.:29.00   3rd Qu.:8.000   3rd Qu.:129.00  
 Max.   :46.60   Max.   :8.000   Max.   :455.0   Max.   :230.00  
     weight      acceleration        year           origin    
 Min.   :1613   Min.   : 8.00   Min.   :70.00   Min.   :1.00   1st Qu.:2225   1st Qu.:13.78   1st Qu.:73.00   1st Qu.:1.00  
 Median :2811   Median :15.50   Median :76.00   Median :1.00  
 Mean   :2978   Mean   :15.54   Mean   :75.96   Mean   :1.57  
 3rd Qu.:17.12   3rd Qu.:79.00   3rd Qu.:2.00  
 Max.   :5140   Max.   :24.80   Max.   :82.00   Max.   :3.00  
\end{verbatim}

\begin{Shaded}
\begin{Highlighting}[]
\FunctionTok{round}\NormalTok{(}\FunctionTok{cor}\NormalTok{(train\_c.df),}\DecValTok{2}\NormalTok{)}
\end{Highlighting}
\end{Shaded}

\begin{verbatim}
               mpg cylinders displacement horsepower weight
mpg           1.00     -0.78        -0.80      -0.78  -0.83
cylinders    -0.78      1.00         0.95       0.84   0.90
displacement -0.80      0.95         1.00       0.90   0.93
horsepower   -0.78      0.84         0.90       1.00   0.86
weight       -0.83      0.90         0.93       0.86   1.00  0.43     -0.51        -0.55      -0.70  -0.43
year          0.57     -0.34        -0.37      -0.41  -0.30
origin        0.57     -0.57        -0.62      -0.46  -0.59
             acceleration  year
mpg                  0.43   0.57
cylinders    -0.34  -0.57
displacement        -0.55 -0.37  -0.62
horsepower          -0.70 -0.41  -0.46
weight       -0.30  -0.59
acceleration         1.00  0.30   0.20
year          1.00   0.18
origin               0.20  0.18   1.00
\end{verbatim}

\begin{Shaded}
\begin{Highlighting}[]
\FunctionTok{round}\NormalTok{(}\FunctionTok{cor}\NormalTok{(test\_c.df),}\DecValTok{2}\NormalTok{)}
\end{Highlighting}
\end{Shaded}

\begin{verbatim}
               mpg cylinders displacement horsepower weight
mpg           1.00     -0.79        -0.81      -0.83  -0.88
cylinders    -0.79      1.00         0.96       0.86   0.93
displacement -0.81      0.96         1.00       0.94   0.95
horsepower   -0.83      0.86         0.94       1.00   0.90
weight       -0.88      0.93         0.95       0.90   1.00
acceleration  0.24     -0.34        -0.42      -0.49  -0.25
year          0.74     -0.37        -0.38      -0.54  -0.44
origin        0.42     -0.61        -0.58      -0.44  -0.52
             acceleration  year origin
mpg                  0.24   0.42
cylinders           -0.34 -0.37  -0.61
displacement        -0.42 -0.38  -0.58
horsepower          -0.49 -0.54  -0.44
weight       -0.44  -0.52
acceleration         1.00  0.11   0.41         0.11  1.00   0.15
origin        0.15   1.00
\end{verbatim}

\begin{Shaded}
\begin{Highlighting}[]

\NormalTok{model1 }\OtherTok{\textless{}{-}} \FunctionTok{lm}\NormalTok{(mpg }\SpecialCharTok{\textasciitilde{}}\NormalTok{ cylinders }\SpecialCharTok{+}\NormalTok{ displacement }\SpecialCharTok{+}\NormalTok{ horsepower }\SpecialCharTok{+}\NormalTok{ weight }\SpecialCharTok{+}\NormalTok{ acceleration }\SpecialCharTok{+}\NormalTok{ year }\SpecialCharTok{+}\NormalTok{ origin, }\AttributeTok{data =}\NormalTok{ train\_c.df)}
\FunctionTok{summary}\NormalTok{(model1)}
\end{Highlighting}
\end{Shaded}

\begin{verbatim}

Call:
lm(formula = mpg ~ cylinders + displacement + horsepower + weight + 
    acceleration + year + origin, data = train_c.df)

Residuals:
    Min      1Q  Median      3Q     Max 
-9.6805 -2.1786 -0.0977  1.9180 13.0364 

Coefficients:
               Estimate Std. Error t value Pr(>|t|)    
(Intercept)  -1.660e+01  4.780e+00  -3.472 0.000578 ***
cylinders    -5.235e-01  3.340e-01  -1.567 0.117947    
displacement  2.042e-02  7.760e-03   2.632 0.008857 ** 
horsepower   -1.750e-02  1.424e-02  -1.229 0.219908    
weight       -6.416e-03  6.785e-04  -9.457  < 2e-16 ***
acceleration  8.742e-02  1.031e-01   0.848 0.396859    
year          7.383e-01  5.259e-02  14.039  < 2e-16 ***
origin        1.516e+00  2.893e-01   5.240 2.73e-07 ***
---
Signif. codes:  0 ‘***’ 0.001 ‘**’ 0.01 ‘*’ 0.05 ‘.’ 0.1 ‘ ’ 1

Residual standard error:3.367  364 degrees of freedom
Multiple R-squared: 0.817,  Adjusted R-squared: 0.8135 
F-statistic: 232.2on 7and 364 DF,  p-value: < 2.2e-16
\end{verbatim}

\begin{Shaded}
\begin{Highlighting}[]
\FunctionTok{print}\NormalTok{(}\FunctionTok{paste0}\NormalTok{(}\StringTok{\textquotesingle{}R{-}sq value is\textquotesingle{}}\NormalTok{ , }\FunctionTok{summary}\NormalTok{(model1)}\SpecialCharTok{$}\NormalTok{r.sq))}
\end{Highlighting}
\end{Shaded}

\begin{verbatim}
[1] "R-sq value is0.817033618528202"
\end{verbatim}

\begin{Shaded}
\begin{Highlighting}[]
\FunctionTok{print}\NormalTok{(}\FunctionTok{paste0}\NormalTok{(}\StringTok{\textquotesingle{}Adjusted R{-}sq value is\textquotesingle{}}\NormalTok{, }\FunctionTok{summary}\NormalTok{(model1)}\SpecialCharTok{$}\NormalTok{adj.r.squared))}
\end{Highlighting}
\end{Shaded}

\begin{verbatim}
[1]
\end{verbatim}

\begin{Shaded}
\begin{Highlighting}[]
\FunctionTok{print}\NormalTok{(}\FunctionTok{paste0}\NormalTok{(}\StringTok{\textquotesingle{}RSE is\textquotesingle{}}\NormalTok{ , }\FunctionTok{summary}\NormalTok{(model1)}\SpecialCharTok{$}\NormalTok{sigma))}
\end{Highlighting}
\end{Shaded}

\begin{verbatim}
[1] "RSE is3.36691839200609"
\end{verbatim}

\begin{Shaded}
\begin{Highlighting}[]
\FunctionTok{print}\NormalTok{(}\FunctionTok{paste0}\NormalTok{(}\StringTok{\textquotesingle{}RMSE is\textquotesingle{}}\NormalTok{ , }\FunctionTok{sqrt}\NormalTok{(}\FunctionTok{mean}\NormalTok{(model1}\SpecialCharTok{$}\NormalTok{residuals}\SpecialCharTok{\^{}}\DecValTok{2}\NormalTok{))))}
\end{Highlighting}
\end{Shaded}

\begin{verbatim}
[1] "RMSE is3.33051820488739"
\end{verbatim}

Taking into account the results, we can see that the model is good but
some of the attributes or predictors are not really useful due to the
fact that really do not influence the final mpg result. \#\#\#
2.1.a.iii)

\begin{Shaded}
\begin{Highlighting}[]
\FunctionTok{ggplot}\NormalTok{(train\_c.df ,}\FunctionTok{aes}\NormalTok{( }\AttributeTok{x =}\NormalTok{ mpg, }\AttributeTok{y =}\NormalTok{ model1}\SpecialCharTok{$}\NormalTok{residuals)) }\SpecialCharTok{+} \FunctionTok{geom\_point}\NormalTok{()}\SpecialCharTok{+} \FunctionTok{labs}\NormalTok{(}\AttributeTok{title =} \StringTok{\textquotesingle{}Residuals graph\textquotesingle{}}\NormalTok{, }\AttributeTok{x =} \StringTok{\textquotesingle{}mpg\textquotesingle{}}\NormalTok{, }\AttributeTok{y =} \StringTok{\textquotesingle{}Residuals\textquotesingle{}}\NormalTok{)}
\end{Highlighting}
\end{Shaded}

\begin{Shaded}
\begin{Highlighting}[]
\ConstantTok{NA}
\end{Highlighting}
\end{Shaded}

\hypertarget{a.iv}{%
\subsubsection{2.1.a.iv)}\label{a.iv}}

\begin{Shaded}
\begin{Highlighting}[]

\FunctionTok{ggplot}\NormalTok{(model1, }\FunctionTok{aes}\NormalTok{( }\AttributeTok{x =}\NormalTok{ model1}\SpecialCharTok{$}\NormalTok{residuals)) }\SpecialCharTok{+} \FunctionTok{geom\_histogram}\NormalTok{(}\AttributeTok{fill =} \StringTok{\textquotesingle{}blue\textquotesingle{}}\NormalTok{, }\AttributeTok{color =} \StringTok{\textquotesingle{}black\textquotesingle{}}\NormalTok{) }\SpecialCharTok{+} \FunctionTok{labs}\NormalTok{(}\AttributeTok{title =} \StringTok{\textquotesingle{}Residuals chart\textquotesingle{}}\NormalTok{, }\AttributeTok{x =} \StringTok{\textquotesingle{}Residuals\textquotesingle{}}\NormalTok{, }\AttributeTok{y =} \StringTok{\textquotesingle{}Probability\textquotesingle{}}\NormalTok{)}
\end{Highlighting}
\end{Shaded}

\begin{Shaded}
\begin{Highlighting}[]
\ConstantTok{NA}
\end{Highlighting}
\end{Shaded}

The residuals do follow a Gaussian distribution clearly.

\begin{enumerate}
\def\labelenumi{\alph{enumi})}
\setcounter{enumi}{1}
\tightlist
\item
\end{enumerate}

\hypertarget{b.i}{%
\subsubsection{2.1.b.i)}\label{b.i}}

\begin{Shaded}
\begin{Highlighting}[]
\FunctionTok{rank}\NormalTok{(}\FunctionTok{summary}\NormalTok{(model1)}\SpecialCharTok{$}\NormalTok{coefficients[,}\StringTok{\textquotesingle{}Pr(\textgreater{}|t|)\textquotesingle{}}\NormalTok{], }\AttributeTok{n =} \DecValTok{3}\NormalTok{) }\SpecialCharTok{\textless{}} \DecValTok{4}
\end{Highlighting}
\end{Shaded}

\begin{verbatim}
 (Intercept)    cylinders displacement   horsepower       weight        FALSE        FALSE        FALSE        FALSE         TRUE 
acceleration         year 
       FALSE         TRUE         TRUE 
\end{verbatim}

Weight, year and origin are the most significant ones. It could also be
drawn from looking at the p values in the aforementioned analysis.

\hypertarget{b.ii}{%
\subsubsection{2.1.b.ii)}\label{b.ii}}

\begin{Shaded}
\begin{Highlighting}[]

\NormalTok{model2 }\OtherTok{\textless{}{-}} \FunctionTok{lm}\NormalTok{(mpg }\SpecialCharTok{\textasciitilde{}}\NormalTok{ weight }\SpecialCharTok{+}\NormalTok{ year}\SpecialCharTok{+}\NormalTok{ origin, }\AttributeTok{data =}\NormalTok{ train.df)}
\FunctionTok{print}\NormalTok{(}\FunctionTok{paste0}\NormalTok{(}\StringTok{\textquotesingle{}R{-}sq value is\textquotesingle{}}\NormalTok{ , }\FunctionTok{summary}\NormalTok{(model2)}\SpecialCharTok{$}\NormalTok{r.sq))}
\end{Highlighting}
\end{Shaded}

\begin{verbatim}
[1] "R-sq value is0.81258059379783"
\end{verbatim}

\begin{Shaded}
\begin{Highlighting}[]
\FunctionTok{print}\NormalTok{(}\FunctionTok{paste0}\NormalTok{(}\StringTok{\textquotesingle{}Adjusted R{-}sq value is\textquotesingle{}}\NormalTok{, }\FunctionTok{summary}\NormalTok{(model2)}\SpecialCharTok{$}\NormalTok{adj.r.squared))}
\end{Highlighting}
\end{Shaded}

\begin{verbatim}
[1] "Adjusted R-sq value is0.811052718203791"
\end{verbatim}

\begin{Shaded}
\begin{Highlighting}[]
\FunctionTok{print}\NormalTok{(}\FunctionTok{paste0}\NormalTok{(}\StringTok{\textquotesingle{}RSE is\textquotesingle{}}\NormalTok{ , }\FunctionTok{summary}\NormalTok{(model2)}\SpecialCharTok{$}\NormalTok{sigma))}
\end{Highlighting}
\end{Shaded}

\begin{verbatim}
[1] "RSE is3.38907360996063"
\end{verbatim}

\begin{Shaded}
\begin{Highlighting}[]
\FunctionTok{print}\NormalTok{(}\FunctionTok{paste0}\NormalTok{(}\StringTok{\textquotesingle{}RMSE is\textquotesingle{}}\NormalTok{ , }\FunctionTok{sqrt}\NormalTok{(}\FunctionTok{mean}\NormalTok{(model2}\SpecialCharTok{$}\NormalTok{residuals}\SpecialCharTok{\^{}}\DecValTok{2}\NormalTok{))))}
\end{Highlighting}
\end{Shaded}

\begin{verbatim}
[1] "RMSE is3.37080353826561"
\end{verbatim}

R-sq value is almost one, so there is a strong linear dependence,
therefore the model is valid.

\hypertarget{b.iii}{%
\subsubsection{2.1.b.iii)}\label{b.iii}}

\begin{Shaded}
\begin{Highlighting}[]
\ConstantTok{NA}
\end{Highlighting}
\end{Shaded}

\begin{Shaded}
\begin{Highlighting}[]
\ConstantTok{NA}
\ConstantTok{NA}
\end{Highlighting}
\end{Shaded}

\hypertarget{b.iv}{%
\subsubsection{2.1.b.iv)}\label{b.iv}}

\begin{Shaded}
\begin{Highlighting}[]
\ConstantTok{NA}
\ConstantTok{NA}
\end{Highlighting}
\end{Shaded}

The residuals of the both models, follow more or less a Gaussian
distribution as it can be seen in both graphs

\hypertarget{b.v}{%
\subsubsection{2.1.b.v)}\label{b.v}}

\begin{Shaded}
\begin{Highlighting}[]

\NormalTok{mod1 }\OtherTok{\textless{}{-}} \FunctionTok{c}\NormalTok{(}\FunctionTok{summary}\NormalTok{(model1)}\SpecialCharTok{$}\NormalTok{r.sq, }\FunctionTok{summary}\NormalTok{(model1)}\SpecialCharTok{$}\NormalTok{adj.r.squared, }\FunctionTok{summary}\NormalTok{(model1)}\SpecialCharTok{$}\NormalTok{sigma, }\FunctionTok{sqrt}\NormalTok{(}\FunctionTok{mean}\NormalTok{(model1}\SpecialCharTok{$}\NormalTok{residuals}\SpecialCharTok{\^{}}\DecValTok{2}\NormalTok{)))}
\NormalTok{mod2 }\OtherTok{\textless{}{-}} \FunctionTok{c}\NormalTok{(}\FunctionTok{summary}\NormalTok{(model2)}\SpecialCharTok{$}\NormalTok{r.sq, }\FunctionTok{summary}\NormalTok{(model2)}\SpecialCharTok{$}\NormalTok{adj.r.squared, }\FunctionTok{summary}\NormalTok{(model2)}\SpecialCharTok{$}\NormalTok{sigma, }\FunctionTok{sqrt}\NormalTok{(}\FunctionTok{mean}\NormalTok{(model2}\SpecialCharTok{$}\NormalTok{residuals}\SpecialCharTok{\^{}}\DecValTok{2}\NormalTok{)))}

\FunctionTok{print}\NormalTok{(mod1)}
\end{Highlighting}
\end{Shaded}

\begin{verbatim}
[1] 0.8170336 0.8135150 3.3669184 3.3305182
[1] 0.8125806 0.8110527 3.3890736 3.3708035
\end{verbatim}

The second model is more cost-efficient due to the fact that perfomrs
less calculations taking into account that uses a lower number of
predictors or attributes. We have get rid of the non-important ones.
Therefore we get the same results with less.

Also p and F values are more important in this second analysis, so it is
more credible.

\begin{enumerate}
\def\labelenumi{\alph{enumi})}
\setcounter{enumi}{2}
\tightlist
\item
\end{enumerate}

\begin{Shaded}
\begin{Highlighting}[]

\FunctionTok{names}\NormalTok{(pred) }\OtherTok{\textless{}{-}} \FunctionTok{c}\NormalTok{(}\StringTok{\textquotesingle{}Confidence\textquotesingle{}}\NormalTok{, }\StringTok{\textquotesingle{}Lower\textquotesingle{}}\NormalTok{, }\StringTok{\textquotesingle{}Upper\textquotesingle{}}\NormalTok{)}
\NormalTok{pred}\SpecialCharTok{$}\NormalTok{Response }\OtherTok{\textless{}{-}}\NormalTok{ test.df[, }\StringTok{\textquotesingle{}mpg\textquotesingle{}}\NormalTok{]}

  \ControlFlowTok{for}\NormalTok{ (x }\ControlFlowTok{in} \DecValTok{1}\SpecialCharTok{:}\FunctionTok{dim}\NormalTok{(pred)[}\DecValTok{1}\NormalTok{])\{}
    \ControlFlowTok{if}\NormalTok{(pred[x, }\StringTok{\textquotesingle{}Lower\textquotesingle{}}\NormalTok{] }\SpecialCharTok{\textless{}}\NormalTok{ pred[x, }\StringTok{\textquotesingle{}Response\textquotesingle{}}\NormalTok{] }\SpecialCharTok{\&}\NormalTok{ pred[x, }\StringTok{\textquotesingle{}Response\textquotesingle{}}\NormalTok{] }\SpecialCharTok{\textless{}}\NormalTok{ pred[x, }\StringTok{\textquotesingle{}Upper\textquotesingle{}}\NormalTok{]) \{}
\NormalTok{      pred[x, }\StringTok{\textquotesingle{}Matches\textquotesingle{}}\NormalTok{] }\OtherTok{=} \DecValTok{1}
\NormalTok{    \} }\ControlFlowTok{else}\NormalTok{\{}
\NormalTok{      pred[x, }\StringTok{\textquotesingle{}Matches\textquotesingle{}}\NormalTok{] }\OtherTok{=} \DecValTok{0}
\NormalTok{    \}}
\NormalTok{  \}  }
\FunctionTok{print}\NormalTok{(pred)}
\end{Highlighting}
\end{Shaded}

\begin{Shaded}
\begin{Highlighting}[]
\NormalTok{number }\OtherTok{=} \FunctionTok{apply}\NormalTok{(pred[}\FunctionTok{c}\NormalTok{(}\StringTok{\textquotesingle{}Matches\textquotesingle{}}\NormalTok{)], }\DecValTok{2}\NormalTok{, sum)}
\FunctionTok{print}\NormalTok{(}\FunctionTok{paste0}\NormalTok{(}\StringTok{\textquotesingle{}Observations predicted: \textquotesingle{}}\NormalTok{, number))}
\end{Highlighting}
\end{Shaded}

\begin{verbatim}
[1] "Observations predicted: 8"
\end{verbatim}

\begin{enumerate}
\def\labelenumi{\alph{enumi})}
\setcounter{enumi}{4}
\tightlist
\item
\end{enumerate}

\begin{Shaded}
\begin{Highlighting}[]

\NormalTok{pred }\OtherTok{\textless{}{-}}  \FunctionTok{data.frame}\NormalTok{(}\FunctionTok{predict}\NormalTok{(model1, test.df, }\AttributeTok{interval =} \StringTok{"prediction"}\NormalTok{, }\AttributeTok{level =} \FloatTok{0.95}\NormalTok{))}
\FunctionTok{names}\NormalTok{(pred) }\OtherTok{\textless{}{-}} \FunctionTok{c}\NormalTok{(}\StringTok{\textquotesingle{}Confidence\textquotesingle{}}\NormalTok{, }\StringTok{\textquotesingle{}Lower\textquotesingle{}}\NormalTok{, }\StringTok{\textquotesingle{}Upper\textquotesingle{}}\NormalTok{)}
\NormalTok{pred}\SpecialCharTok{$}\NormalTok{Response }\OtherTok{\textless{}{-}}\NormalTok{ test.df[, }\StringTok{\textquotesingle{}mpg\textquotesingle{}}\NormalTok{]}

  \ControlFlowTok{for}\NormalTok{ (x }\ControlFlowTok{in} \DecValTok{1}\SpecialCharTok{:}\FunctionTok{dim}\NormalTok{(pred)[}\DecValTok{1}\NormalTok{])\{}
    \ControlFlowTok{if}\NormalTok{(pred[x, }\StringTok{\textquotesingle{}Lower\textquotesingle{}}\NormalTok{] }\SpecialCharTok{\textless{}}\NormalTok{ pred[x, }\StringTok{\textquotesingle{}Response\textquotesingle{}}\NormalTok{] }\SpecialCharTok{\&}\NormalTok{ pred[x, }\StringTok{\textquotesingle{}Response\textquotesingle{}}\NormalTok{] }\SpecialCharTok{\textless{}}\NormalTok{ pred[x, }\StringTok{\textquotesingle{}Upper\textquotesingle{}}\NormalTok{]) \{}
\NormalTok{      pred[x, }\StringTok{\textquotesingle{}Matches\textquotesingle{}}\NormalTok{] }\OtherTok{=} \DecValTok{1}
\NormalTok{    \} }\ControlFlowTok{else}\NormalTok{\{}
\NormalTok{      pred[x, }\StringTok{\textquotesingle{}Matches\textquotesingle{}}\NormalTok{] }\OtherTok{=} \DecValTok{0}
\NormalTok{    \}}
\NormalTok{  \}  }
\FunctionTok{print}\NormalTok{(pred)}
\NormalTok{number }\OtherTok{=} \FunctionTok{apply}\NormalTok{(pred[}\FunctionTok{c}\NormalTok{(}\StringTok{\textquotesingle{}Matches\textquotesingle{}}\NormalTok{)], }\DecValTok{2}\NormalTok{, sum)}
\FunctionTok{print}\NormalTok{(}\FunctionTok{paste0}\NormalTok{(}\StringTok{\textquotesingle{}Observations predicted: \textquotesingle{}}\NormalTok{, number))}

 
\end{Highlighting}
\end{Shaded}

\hypertarget{part-2.1.f}{%
\subsubsection{Part 2.1.f}\label{part-2.1.f}}

The confidence interval defines to which stand do we accept the errors.
That is to say, the limit f that we accept will only be defined by the
confidence interval that we choose.

In the `e' one, we chose the confidence interval, whereas in the `f'
one, we choose the prediction interval. As it will be explained later,
the predicton interval it is usually wider than the confidence one. That
is why in the prediction interval we got 20 observations predicted
within the interval. It is less restrictive the analysis.

\hypertarget{part-2.1.f.i}{%
\subsubsection{Part 2.1.f.i}\label{part-2.1.f.i}}

The f one, with the prediction interval got more many matches, 12 more.

\hypertarget{part-2.1.f.ii}{%
\subsubsection{Part 2.1.f.ii}\label{part-2.1.f.ii}}

It happens because the prediction interval is usually bigger than the
confidence interval due to the fact that it contains more types of
errors. For example, compared to the confidence interval, it also
contains the reducible error appart from the irreducible.

\hypertarget{references}{%
\subsubsection{References}\label{references}}

Multiple regression model

\url{https://www.analyticsvidhya.com/blog/2020/12/predicting-using-linear-regression-in-r/}

\end{document}
